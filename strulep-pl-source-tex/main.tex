	
\documentclass[10pt]{article}
\usepackage{geometry}
\geometry{a4paper, landscape}
\geometry{left=0.5cm,right=2cm,top=1cm,bottom=1cm}
% The paracol package lets you typeset columns of text in parallel
\usepackage{paracol}
\usepackage{booktabs}
%polish support
\usepackage[T1]{fontenc}
%load colors package
%colors list at https://www.overleaf.com/learn/latex/Using_colours_in_LaTeX
\usepackage[dvipsnames]{xcolor}

%define row text colors

\definecolor{rowcolor}{HTML}{3f3f3f}
\definecolor{headcolor}{HTML}{262dc1}


% If using xelatex or lualatex:
%\setmainfont{Roboto Slab}
%\setsansfont{Lato}

% If using pdflatex:
\usepackage[rm]{roboto}
\usepackage[defaultsans]{lato}

\renewcommand{\familydefault}{\sfdefault}

\usepackage{array}
\newcommand{\PreserveBackslash}[1]{\let\temp=\\#1\let\\=\temp}
\newcolumntype{C}[1]{>{\PreserveBackslash\centering}p{#1}}
\newcolumntype{R}[1]{>{\PreserveBackslash\raggedleft}p{#1}}
\newcolumntype{L}[1]{>{\PreserveBackslash\raggedright}p{#1}}

\usepackage{graphicx}
\graphicspath{ {./img/} }

%Setup column widths in tables
\newlength{\firstcolumnwidth}
\newlength{\secondcolumnwidth}
\setlength{\firstcolumnwidth}{0.7cm}
\setlength{\secondcolumnwidth}{5cm}

%Setup table rule widths
\newlength{\topandbotrulewidth}
\newlength{\midrulewidth}
\setlength{\topandbotrulewidth}{1.2pt}
\setlength{\midrulewidth}{0.3pt}

\setlength{\columnsep}{0.5cm}



%commands for tables headers and rows
\newcommand{\tableheader}[1]
{	\multicolumn{2}{l}{\textcolor{headcolor}{\bfseries{#1}}}\\
	\toprule[\topandbotrulewidth]
	\emph{skrót} & \emph{opis}\\
	\toprule[\topandbotrulewidth]
}
%insert row
\newcommand{\addNewTableRow}[2]
{	{\textcolor{rowcolor} {\footnotesize{#1}}} & {\textcolor{rowcolor} {\footnotesize{#2}}}\\
	\midrule[\midrulewidth]
}






\begin{document}
	\columnratio{0.08}
	
	% Start a 2-column paracol. Both the left and right columns will automatically
	% break across pages if things get too long.
	
	\begin{paracol}{2}
		\includegraphics[scale=.15]{krei_main_logo_vector.pdf}
		\switchcolumn
		{\large Piotr Zieliński}\\
	\end{paracol}
	\bigskip
	%% Set the left/right column width ratio to 6:4.
	\columnratio{0.25}
	% Start a 2-column paracol. Both the left and right columns will automatically
	% break across pages if things get too long.
	
	\begin{paracol}{4}
		\renewcommand{\arraystretch}{0.8}% Tighter
		%first column start here
		\begin{tabular}{L{\firstcolumnwidth} L{\secondcolumnwidth}}
			\tableheader{FUNDAMENTY}
			\addNewTableRow{STF}{stopa fundamentowa}
			\addNewTableRow{LWA}{ława fundamentowa}
			\addNewTableRow{PFU}{płyta fundamentowa}
			\addNewTableRow{PNE}{przegłębienie w płycie}
			\addNewTableRow{PGR}{pogrubienie w płycie}
			\addNewTableRow{OPR}{mur oporowy}
			\addNewTableRow{OCF}{oczep - stopa na palach}
			\addNewTableRow{RFU}{ruszt fundamentowy}
			\addNewTableRow{POD}{podwalina}
			\addNewTableRow{STD}{studnia fundamentowa}
			\addNewTableRow{KSN}{keson}
			\addNewTableRow{SKR}{skrzynia fundamentowa}
			\addNewTableRow{OCL}{oczep - ława na palach}
		\end{tabular}
		\bigskip
		
		\begin{tabular}{L{\firstcolumnwidth} L{\secondcolumnwidth}}
			\tableheader{OBUDOWY WYKOPÓW}
			\addNewTableRow{SSZ}{ściany szczelinowe}
			\addNewTableRow{BER}{obudowa berlińska}
			\addNewTableRow{PDA}{palisada}
			\addNewTableRow{LAR}{Larsen / ścianka szczelna}
			\addNewTableRow{RZP}{rozpora}
			\addNewTableRow{KTW}{kotwa gruntowa}
			\addNewTableRow{OCP}{oczep obudowy wykopu}
			\addNewTableRow{MPR}{murek prowadzący}
%			\addNewTableRow{PNA}{podwalina}
%			\addNewTableRow{STD}{studnia fundamentowa}
%			\addNewTableRow{KSN}{keson}
%			\addNewTableRow{SKR}{skrzynia fundamentowa}
		\end{tabular}
		\bigskip
		
		%Switch to second column
		\switchcolumn
		
		\begin{tabular}{L{\firstcolumnwidth} L{\secondcolumnwidth}}
			\tableheader{BUDYNEK - \emph{elementy nośne pionowe}}
			\addNewTableRow{SUP}{słup}
			\addNewTableRow{TRZ}{trzpień}
			\addNewTableRow{FIL}{filarek}
			\addNewTableRow{SCN}{ściana nośna murowana}
			\addNewTableRow{SCZ}{ściana nośna żelbetowa}
			\addNewTableRow{SCW}{ściana nienośna - wypełniająca}
			\addNewTableRow{TAR}{tarcza żelbetowa}
		\end{tabular}
		\bigskip

		\begin{tabular}{L{\firstcolumnwidth} L{\secondcolumnwidth}}
			\tableheader{BUDYNEK - \emph{elementy nośne poziome}}
			\addNewTableRow{NPR}{nadproże prefabrykowane}
			\addNewTableRow{BLK}{belka monolityczna}
			\addNewTableRow{POD}{podciąg żelbetowy}
			\addNewTableRow{NAD}{nadciąg / belka nad stopem}
			\addNewTableRow{ZBR}{żebro stropu}
			\addNewTableRow{STR}{strop żelbetowy}
			\addNewTableRow{SFI}{strop typu filigran}
			\addNewTableRow{PFI}{płyta filigran}
			\addNewTableRow{TER}{strop typu terriva}
			\addNewTableRow{SKN}{strop z płyt kanałowych}
			\addNewTableRow{BKN}{balkon}
			\addNewTableRow{WSP}{wspornik}
			\addNewTableRow{WNC}{wieniec monolityczny}
		\end{tabular}
		\bigskip

		%Switch to third column
		\switchcolumn

		%Switch to fourth column
		\switchcolumn

	\end{paracol}


\end{document}